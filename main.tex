\documentclass{ELSAbeamer}
\usepackage[utf8]{inputenc}

\usepackage{lipsum}
\usepackage{listings}

\title[ELSAbeamer]{How to Use the ELSAbeamer \LaTeX{} Class}
\author[K.-Y. Chang]{Kuan-Yu Chang}
\date[2021/06/14]{June 14, 2021}

\begin{document}

\makecover

\section{Lists}

\begin{frame}{Unordered Lists}
\begin{columns}
\column{.5\linewidth}
\begin{itemize}
    \item item1
    \begin{itemize}
        \item item1-1
        \item item1-2
        \item item1-3
    \end{itemize}
    \item item2
    \begin{itemize}
        \item item2-1
        \item item2-2
        \item item2-3
    \end{itemize}
    \item item3
    \begin{itemize}
        \item item3-1
        \item item3-2
        \item item3-3
    \end{itemize}
\end{itemize}

\column{.5\linewidth}
\begin{itemize} \itemsep2em
    \item item4
    \begin{itemize} \itemsep1em
        \item item4-1
        \item item4-2
        \item item4-3
    \end{itemize}
    \item item5
    \begin{itemize} \itemsep1em
        \item item5-1
        \item item5-2
        \item item5-3
    \end{itemize}
\end{itemize}
\end{columns}
\end{frame}

\begin{frame}{Ordered lists}
\begin{columns}
\column{.5\linewidth}
\begin{enumerate}
    \item item1
    \begin{enumerate}
        \item item1-1
        \item item1-2
        \item item1-3
    \end{enumerate}
    \item item2
    \begin{enumerate}
        \item item2-1
        \item item2-2
        \item item2-3
    \end{enumerate}
    \item item3
    \begin{enumerate}
        \item item3-1
        \item item3-2
        \item item3-3
    \end{enumerate}
\end{enumerate}

\column{.5\linewidth}
\begin{enumerate} \itemsep2em
    \setcounter{enumi}{3}
    \item item4
    \begin{enumerate} \itemsep1em
        \item item4-1
        \item item4-2
        \item item4-3
    \end{enumerate}
    \item item5
    \begin{enumerate} \itemsep1em
        \item item5-1
        \item item5-2
        \item item5-3
    \end{enumerate}
\end{enumerate}
\end{columns}
\end{frame}

\section{Structuring Elements}

\begin{frame}{Text blocks}
\framesubtitle{In plain, example, and \alert{alert} flavour}
\alert{This text} is highlighted.

\begin{block}{A plain block}
    This is a plain block containing some \alert{highlighted text}.
\end{block}

\begin{exampleblock}{An example block}
    This is an example block containing some \alert{highlighted text}.
\end{exampleblock}

\begin{alertblock}{An alert block}
    This is an alert block containing some \alert{highlighted text}.
\end{alertblock}
\end{frame}

\begin{frame}{Definitions, theorems, and proofs}
\framesubtitle{All integers divide zero}
\begin{definition}
    $\forall a,b\in\mathbb{Z}: a\mid b\iff\exists c\in\mathbb{Z}:a\cdot c=b$
\end{definition}

\begin{theorem}
    $\forall a\in\mathbb{Z}: a\mid 0$
\end{theorem}

\begin{proof}
    $\forall a\in\mathbb{Z}: a\cdot 0=0$
\end{proof}
\end{frame}

\section{Numerals and Mathematics}

\begin{frame}{Numerals and Mathematics}
\framesubtitle{Formulae, equations, and expressions}
\begin{columns}
\column{.20\linewidth}
1234567890

\column{.20\linewidth}
\oldstylenums{1234567890}

\column{.25\linewidth}
$\hat{x}$, $\check{x}$, $\tilde{a}$, $\bar{a}$, $\dot{y}$, $\ddot{y}$

\column{.35\linewidth}
$\iint f(x,y,z)\,\mathsf{d}x\mathsf{d}y\mathsf{d}z$
\end{columns}

\begin{columns}
\column{.5\linewidth}
\begin{equation*}
    \frac{1}{\displaystyle 1+
            \frac{1}{\displaystyle 2+
            \frac{1}{\displaystyle 3+x}}} +
            \frac{1}{1+\frac{1}{2+\frac{1}{3+x}}}
\end{equation*}

\column{.5\linewidth}
\begin{equation*}
    F:\left| \begin{array}{ccc}
          F''_{xx} & F''_{xy} &  F'_x \\
          F''_{yx} & F''_{yy} &  F'_y \\
          F'_x     & F'_y     & 0
          \end{array}\right| = 0
\end{equation*}
\end{columns}

\begin{columns}
\column{.3\linewidth}
\begin{equation*}
    \iint_{\mathbf{x} \in \mathbb{R}^2} \langle \mathbf{x},\mathbf{y}\rangle\,\mathsf{d}\mathbf{x}
\end{equation*}

\column{.33\linewidth}
\begin{equation*}
    \overline{\overline{a\alpha}^2+\underline{b\beta}
          +\overline{\overline{d\delta}}}
\end{equation*}

\column{.37\linewidth}
\begin{equation*}
    \left] 0,1\right[ + \lceil x \rfloor - \langle x,y\rangle
\end{equation*}
\end{columns}

\begin{columns}
\column{.4\linewidth}
\begin{eqnarray*}
    e^x &\approx& 1+x+x^2/2! + \\
        && {}+x^3/3! + x^4/4!
\end{eqnarray*}

\column{.6\linewidth}
\begin{equation*}
    \binom{n+1}{k} = \binom{n}{k} + \binom{n}{k-1}
\end{equation*}
\end{columns}
\end{frame}

\section{Tables, Figures, and Code listlings}

\subsection{Tables}

\begin{frame}{An example table}
\begin{table}[t]
    \begin{tabular}{ccr}
        \toprule
        First Name & Last Name & Date of Birth \\
        \midrule
        John    & Doe       & 3/12/1920 \\
        Peter   & Smith     & 6/5/1967 \\
        Julia   & Jones     & 9/26/1977 \\
        Jane    & Miller    & 10/5/1966 \\
        Peter   & Smith     & 1/3/1901 \\
        \bottomrule
    \end{tabular}
    \caption{Personal data.}
\end{table}
\end{frame}

\subsection{Figures}

\begin{frame}{Dummy Text}
\begin{columns}
\column{.6\linewidth} \justifying
\scriptsize \lipsum[1]

\column{.4\linewidth}
\begin{figure}
    \centering
     \includegraphics[width=.8\linewidth]{example-image-a}
     \caption{An example image.}
\end{figure}
\end{columns}
\end{frame}

\subsection{Code listlings}

\defverbatim[colored]\sleepSort{
\begin{lstlisting}[language=C,tabsize=2]
#include <stdio.h>
#include <unistd.h>
#include <sys/types.h>
#include <sys/wait.h>

int main(int argc, char **argv) {
    while (--c > 1 && !fork());
    sleep(c = atoi(v[c]));
    printf("%d\n", c);
    wait(0);
    return 0;
}
\end{lstlisting}}
\begin{frame}{An example source code in C}
\sleepSort
\end{frame}

\section{Citations and Bibliography}

\begin{frame}{\TeX, \LaTeX, and Beamer} \justifying
\TeX\ is a programming language for the typesetting of documents. It was created by Donald Erwin Knuth in the late 1970s and it is documented in \emph{The \TeX book}~\cite{knuth1984texbook}.

\bigskip

In the early 1980s, Leslie Lamport created the initial version of \LaTeX, a high-level language on top of \TeX, which is documented in \emph{\LaTeX{}: A Document Preparation System}~\cite{lamport1994latex}. There exists a healthy ecosystem of packages that extend the base functionality of \LaTeX; \emph{The \LaTeX{} Companion}~\cite{mittelbach2004latex} acts as a guide through the ecosystem.

\bigskip

In 2003, Till Tantau created the initial version of Beamer, a \LaTeX{} package for the creation of presentations. Beamer is documented in the \emph{User's Guide to the Beamer Class}~\cite{tantau2004user}.
\end{frame}

\makebib{references}

\end{document}
